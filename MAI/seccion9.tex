\documentclass[12pt,letterpaper]{article}
\usepackage[utf8]{inputenc}
\usepackage[spanish]{babel}
\usepackage{amsmath}
\usepackage{amsfonts}
\usepackage{amssymb}
\usepackage{graphicx}
\usepackage{multicol}
\usepackage{enumitem}
\usepackage{ragged2e}

\usepackage[left=2cm,right=2cm,top=2cm,bottom=4cm]{geometry}
\author{Javier Said Naranjo Miranda}
\begin{document}
\subsection* {Cap\'itulo 9 Funciones Complejas}
	\subsubsection* {Secci\'on 9.1 Limites, continuidad, derivadas}
	\subsection* {Problemas Secci\'on 9.1}
\justify
En cada problema del 1 al 16, encuentre $u$ y $v$ de manera que $f(z) = u(x,y) + iv(x,y)$, determine todos los puntos(si los hay)en donde las ecuaciones de Cauchy-Riemann se satisfacen, y determine todos los puntos en donde la funci\'on es diferenciable. Se pueden suponer todos los resultados familiares de la continuidad de funciones reales de dos variables.
\begin{enumerate}
\item $f(z)= z-i$ \\
Soluci\'on.\\
$f(z)=x+iy-i$\\
$f(z)=x+i(y-1)$\\
$u=x$ \hspace{1cm} $v=y-1$\\
Condiciones de Cauchy-Riemann.\\
$u_x=v_y$ \hspace{1cm} $u_y=-v_x$\\
$u_x=1=v_y$\\
$u_y=0=-v_x$\\
Las condiciones de Cauchy-Riemann se satisfacen $\forall z \epsilon \mathbb{C}$\\
Al cumplirse las condiciones de C-R la funci\'on es diferenciable en todo punto.
\item $f(z)= |z|$\\
Soluci\'on.\\
$f(z)=|x+iy|$\\
$f(z)=\sqrt{x^2+y^2}$\\
$u=\sqrt{x^2+y^2} \hspace{1cm} v=0$\\
Condiciones de Cauchy-Riemann.\\
$u_x=v_y$ \hspace{1cm} $u_y=-v_x$\\
$u_x=\frac{1}{2}(x^2+y^2)^{-\frac{1}{2}}(2x)\neq 0=v_y$\\
$u_y=\frac{1}{2}(x^2+y^2)^{-\frac{1}{2}}(2y)\neq 0=-v_x$\\
Las condiciones de Cauchy-Riemann no se satisfacen en ning\'un punto\\
Al no cumplirse las condiciones de C-R la funci\'on no es diferenciable en ning\'un punto.
\item $f(z)= i|z|^2$
Soluci\'on.\\
$f(z)=i(\sqrt{x^2+y^2})^2$\\
$f(z)=i(x^2+y^2)$\\
$u=0 \hspace{1cm} v=x^2+y^2$\\
Condiciones de Cauchy-Riemann.\\
$u_x=v_y$ \hspace{1cm} $u_y=-v_x$\\
$u_x = 0 \neq 2y = v_y$\\
$u_y = 0 \neq -2x =-v_x$\\
Las condiciones de Cauchy-Riemann solo se satisfacen si x=0 , y=0 \\
Al no cumplirse las condiciones de C-R $\forall z \epsilon \mathbb{C}$ y no ser anal\'itica, la funci\'on no es diferenciable en ning\'un punto.
\item $f(z)= \frac{z}{Re(z)}$\\
Soluci\'on.\\
$f(z)=\frac{x+iy}{x}$\\
$u=1 \hspace{1cm} v=\frac{y}{x}$\\
Condiciones de Cauchy-Riemann.\\
$u_x=v_y$ \hspace{1cm} $u_y=-v_x$\\
$u_x = 0 \neq \frac{1}{x} = v_y$\\
$u_y = 0 \neq \frac{y}{x^2} =-v_x$\\
Las condiciones de Cauchy-Riemann no se satisfacen en ning\'un punto. \\
Al no cumplirse las condiciones de C-R la funci\'on no es diferenciable en ning\'un punto.
\item $f(z)= \overline{z}^2$\\
Soluci\'on.\\
$f(z)=(x-iy)^2$\\
$f(z)=x^2-2ixy-y^2$\\
$u=x^2-y^2 \hspace{1cm} v=-2xy$\\
Condiciones de Cauchy-Riemann.\\
$u_x=v_y$ \hspace{1cm} $u_y=-v_x$\\
$u_x = 2x \neq -2x = v_y$\\
$u_y = -2y \neq 2y =-v_x$\\
Las condiciones de Cauchy-Riemann solo se satisfacen si x=0, y=0. \\
Al no cumplirse las condiciones de C-R $\forall z \epsilon \mathbb{C}$ y no ser anal\'itica, la funci\'on no es diferenciable en ning\'un punto.
\item $f(z)= -4z + \frac{1}{z}$\\
Soluci\'on.\\
$f(z)=-4(x+iy) + \frac{1}{x+iy}(\frac{x-iy}{x-iy})$\\
$f(z)=-4x-4iy + \frac{x}{x^2+y^2}-\frac{iy}{x^2+y^2}$\\
$u=-4x+\frac{x}{x^2+y^2} \hspace{1cm} v=-4y-\frac{y}{x^2y^2}$\\
Condiciones de Cauchy-Riemann.\\
$u_x=v_y$ \hspace{1cm} $u_y=-v_x$\\
$u_x =-4+\frac{x^2+y^2-2x^2}{(x^2+y^2)^2}= -4+\frac{y^2-x^2}{(x^2+y^2)^2}= -4-\frac{x^2+y^2+2y^2}{(x^2+y^2)^2} = v_y$\\
$u_y = \frac{-2xy}{(x^2+y^2)^2}=-v_x$\\
Las condiciones de Cauchy-Riemann se satisfacen $\forall z \epsilon \mathbb{C}-\{0\}$\\
Al cumplirse las condiciones de C-R $\forall z \epsilon \mathbb{C}-\{0\}$, la funci\'on es diferenciable $\forall z \epsilon \mathbb{C}-\{0\}$.
\item $f(z)= Re(z)-Im(z)$
Soluci\'on\\
$f(z)= x-y$\\
$u=x-y \hspace{1cm} v=0$\\
Condiciones de Cauchy-Riemann.\\
$u_x=v_y$ \hspace{1cm} $u_y=-v_x$\\
$u_x = 1 \neq 0 = v_y$\\
$u_y = -1\neq 0 =-v_x$\\
Las condiciones de Cauchy-Riemann no se satisfacen en ning\'un punto.\\
Al no cumplirse las condiciones de C-R en ning\'un punto, la funci\'on no es diferenciable en ning\'un punto.
\item $f(z)= Im(\frac{2z}{z+1})$ \\
Soluci\'on\\ \\
$\frac{2z}{z+1} = \frac{2(x+iy)}{x+iy+1} = \frac{2x+2iy}{(x+1)+iy} (\frac{(x+1)-iy}{(x+1)-iy})= \frac{(2x+2iy)((x+1)-iy)}{(x+1)^2+y^2} = \frac{2x(x+1)-2ixy+2iy(x+1)+2y^2}{(x+1)^2+y^2}$\\
$f(z)= Im(\frac{2x(x+1)-2ixy+2iy(x+1)+2y^2}{(x+1)^2+y^2})$\\
$f(z)= \frac{-2xy+2xy+2y}{(x+1)^2+y^2}$\\
$f(z)= \frac{2y}{(x+1)^2+y^2}$\\
$u=\frac{2y}{(x+1)^2+y^2} \hspace{1cm} v=0$\\
Condiciones de Cauchy-Riemann.\\
$u_x=v_y$ \hspace{1cm} $u_y=-v_x$\\ \\
$u_x=\frac{-2y(2)(x+1)}{((x+1)^2+y^2)^2} = \frac{-4xy-4y}{((x+1)^2+y^2)^2} \neq 0 = v_y$\\ \\
$u_y = \frac{(2)((x+1)^2+y^2)-(2y)(2y)}{((x+1)^2+y^2)^2} = \frac{2x^2+4x+2+2y^2-4y^2}{((x+1)^2+y^2)^2} = \frac{2x^2+4x-2y^2+2}{((x+1)^2+y^2)^2} \neq 0 =-v_x$\\ \\
Las condiciones de Cauchy-Riemann no se satisfacen en ning\'un punto.\\
Al no cumplirse las condiciones de C-R la funci\'on no es diferenciable en ning\'un punto.
\end{enumerate}
	\subsubsection* {Secci\'on 9.2 Series de potencias}
	\subsection* {Problemas Secci\'on 9.2}
\justify
En cada uno de los problemas, determine el radio de convergencia y el disco abierto de convergencia de la serie de potencias.
\begin{enumerate}
\item $\sum\limits_{n=0}^\infty\frac{n+1}{2^n}(z+3i)^n$
\item $\sum\limits_{n=0}^\infty\frac{n^n}{(n+1)^n}(z-1+3i)^n$
\item $\sum\limits_{n=0}^\infty\frac{i^n}{2^{n+1}}(z+8i)^n$
\item $\sum\limits_{n=0}^\infty\frac{n^2}{2n+1}(z+6+2i)^n$
\item $\sum\limits_{n=0}^\infty\frac{e^{in}}{2n+1}(z+4)^n$
\item ?`Es posible que $\sum c_{n}(z-2i)^n$ converja en 0 y diverja en $i$?
\item Suponga $\sum\limits_{n=0}^\infty c_{n}z^n$ tiene radio de convergencia $R$ y $\sum\limits_{n=0}^\infty c_{n}^*z^n$ tiene radio de convergencia $R^*$. ?`Qu\'e se puede decir acerca del radio de convergencia de $\sum\limits_{n=0}^\infty(c_{n}+c_{n}^*)z^n$?
\item Considere $\sum\limits_{n=0}^\infty c_{n}z^n$, donde $c_{n}=2$  si $n$ es par y $c_{n}=1$ si $n$ es impar. Pruebe que el radio de convergencia de esta serie de potencias es 1, pero que este n\'umero no se puede calcular usando el criterio de la raz\'on.(Esto significa simplemente que no siempre se puede usar este criterio para determinar el radio de convergencia de una serie de potencias.)
\end{enumerate}

\end{document}
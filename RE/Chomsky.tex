\documentclass[12pt,letterpaper]{article}
\usepackage[utf8]{inputenc}
\usepackage[spanish]{babel}
\usepackage{amsmath}
\usepackage{amsfonts}
\usepackage{amssymb}
\usepackage{graphicx}
\usepackage{ragged2e}
\usepackage{cite}
\usepackage{float}
\usepackage{wasysym}
\usepackage{subfig}
\graphicspath{ {imagenes/} }
\usepackage{amsmath, amsfonts, amssymb}
\usepackage[left=2.5cm,right=2.5cm,top=2cm,bottom=2cm]{geometry}
\author{Javier Said Naranjo Miranda\\ Grupo: 2CM4}
\title{Teor\'ia Computacional\\ Gram\'aticaslibres de contexto}
\date{7 de noviembre de 2016}
\begin{document}
\maketitle
\justify
Se crear\'a una cadena con 10 sentencias if-then-else, mediante una gram\'atica libre de contexto.\\

Esta tendr\'a la estructura:\\ \\
<statement>::=if<condition>then<statement>[[¨[
[;else <statement>] \\

Ademas se usaran las siguientes reglas:\\
$S \rightarrow iCtSA$
$A \rightarrow ;eS$

\\

\textbf{Construcci\'on de la gram\'atica.}\\

\end{document}